%!TEX root = ../Chapter3.tex
\section{Data, information and knowledge}

At this point it could be useful to consider for a moment some relevant concepts for decision making and try to understand if and how they change in the context of an environment as the one described so far. Data, information and knowledge are key notions underlying every decision, but what do they mean? They are sometimes erroneously used exchangeably, but they indeed refer to completely different things. Some more precise definition follow.

Data can be seen as external stimuli, facts that can be analyzed or used in an effort to gain knowledge or make decisions. They can be structured or unstructured, noisy or clean, relevant or irrelevant.

Information refers to data that has been given some meaning by way of relational connection. The meaning applied to the data may not necessarily be useful. It is important to notice that without information one can not have knowledge.

Knowledge is the concise and appropriate collection of information in a way that makes it useful. Knowledge refers to a deterministic process where patterns within a given set of information are ascertained. It has some useful and applicable uses.

In other words, information is the meaning given to data by the knower according to an interpretative frame; knowledge is the conceptual frame or scheme through which data are interpreted and given meaning becoming information.

It is clear now how the meaning of data depends by the perspective of the inquirer, thus generating ambiguity. Different people could infer different things from the same data and build their own, perhaps conflicting, knowledge.

\section{The perspective on the environment}
If knowledge is nothing but a personal view on the data provided by the environment, then the latter can be considered as an actor interacting in a network. This idea was deeply analyzed by Callon \cite{10} and led him to formulate the actor network theory. This theory highlighted how sometimes the relationships between the actors of a network are even more important than the actors themselves and that the environment in which the network grows is not a passive element, the background of the scene, but a node of the network, perhaps one of the most influent. The environment becomes a central node and all the others should consider its influence when making decisions. It can be an ally or an enemy, one may have to exploit it or to overwhelm it. In any case its role changes what it is understood and how decisions are made.

\section{Agnosticism and social identity}
It is clear that knowledge depends on one’s perspective and that this perspective depends, in turn, on the environment. But what kind of knowledge can one get from the ambiguous environment that is being considered?

The answer is pretty simple: nothing. If nothing can be undoubtedly stated, then nothing can be undoubtedly known. Such situation is generally referred to as agnosticism, the condition of who does not know, or believes that it is impossible to know, if a god exists or, broadly, anything \cite{11} one can think of.

This situation obviously has a great impact also on the social identity, the portion of an individual's self-concept derived from perceived membership in a relevant social group. As originally formulated by Tajfel and Turner in the 1970s and the 1980s \cite{12} social identity theory is described as a theory that predicts certain intergroup behaviours on the basis of perceived group status differences, the perceived legitimacy and stability of those status differences, and the perceived ability to move from one group to another. The term social is suggested for describing the joint contributions of both social identity theory and self-categorization theory.

If this agnosticism redefines the way people interact with the others and how they behave in an active environment, then how decisions are made?
