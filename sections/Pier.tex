%!TEX root = ../Chapter3.tex
\section{Introduction}
In the previous chapter the multiple options available to people when the knowledge they can acquire from the outside world is objective and identical, have been deeply analyzed. In these cases the environment is given and can be freely inquired, so probabilities of interesting events can be calculated, yielding cases of certainty, risk-based decisions or choices founded on a trade-off between acquiring more information and the cost of such researches.

But what happens if the outside environment is an abstraction and cannot be accessed? Shifting to this new environment a clear distinction between exogenous and endogenous probability has to be made. A probability is said to be exogenous if it is \textit{given from out of there}, meaning it has an external cause or origin. In the ambiguous environment, on the contrary, probabilities are endogenous, which means they are \textit{coming out of us}, and they are often based on someone’s beliefs.

In this chapter the endogenous field of \textit{weak} ambiguity is presented and answers to some important questions are given. Are decisions really influenced by knowledge? What can be identified as knowledge? How can it be acquired? More information does not always bring to better decisions and an excess can cause confusion. Furthermore, sometimes decisions cannot be made just after an information search process, because probabilities are inside us, not externally given. Making decisions in an ambiguous environment is not easy, it is a matter of perspectives: there is not a unique rational point of view to follow during the decision making process.
Knowledge is based on a personal interpretation and since everyone can interpret facts in different ways, one can not decide who is right and who is not. An example of the ambiguous environment can be found in politics. A politician takes decisions based on the vision of his political party, but changing party can lead him to change also his decisions. As said before, it is a matter of perspectives.

Is it possible to destroy the reality and make new innovative decisions in an ambiguous environment ruled by perspectives and common sense? For example, imagine a completely empty room destined to a lecture. Students meet each other there for the class and everyone is going to do routine actions: stand up, talk, pay attention, play with phones, and so on and so forth. Now imagine the same scenario set in a room where chairs are divided into two groups: on the left side of the room chairs are sparse, whereas on the right side they are closed together. What would students do before the teacher comes? Commonly, people would put all the chairs in order, making a square or a circle, or just arranging them like in the right side, according to the \textit{idea of class}. Is this an innovative decision? Through this chapter a step forward is taken into innovation and new decisional models and strategies are suggested. The aim is to teach how to spread all chairs in the room, upside down, creating a stimulating disorder that brings fresh original ideas, something that does not fit in standards.

The first paragraph of this chapter examines in depth the \textit{viewpoint} concept, based on the theory of Paul Feyerabend on \textit{what is True and what seems to be True} and the definition of aesthetics by Kuhn, that express how what really matters is what looks nice according to us.

Moving on the meaning of Data, Knowledge and Information, the actor network theory formulated by Callon is analyzed, with a focus on agnosticism.

The core of the chapter presents the art of decisions making in an ambiguous environment explained by March, with a full description of a decisional model, the \textit{garbage can}, and the relative organizational model, called \textit{spaghetti organization}. The chapter ends with conclusions, focused on the role of innovation and entrepreneurship.
