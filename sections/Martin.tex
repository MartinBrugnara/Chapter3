%!TEX root = ../Chapter3.tex
\section{Feyerabend relativism}

According to Paul Feyerabend \textit{\textit{et al.}}\cite{PF} the discussion about what is true
and what it seems to be true lasts from the Pre-Socratic philosopher age.
On this topic, F. \textit{\textit{et al.}} assert that ``true is not true, true is whatever is true for
a single being; each one could have a different opinion about the truth, relative
to themselves: to their experience, to their dreams, to their being.''

Thus the \textit{relativeness} of the definition of truth directly affects the
understanding of reality, the ruling and living. This usually  leads to
an averaged vision of the personal opinionated views called \textit{common sense}.
Once the \textit{common sense} is defined and accepted by the whole community,
it influences the interpretation of the truth for the future actions.
Sometimes this influence is so important that leads to a reconsideration of the
past.\footnote{If ``the history is written by the winners'', then these winners can impose their will and ideas, shaping the common sense of future generation.}
Thus \textit{the truth} can be considered both the \textit{result} of the
process that tries to understand and model the world, both an
\textit{instrument} to modify and control it.

The relativism is essentially the opposite of the objectivism, which assert that
everybody, regardless of their own perception and opinions, lives in the reality.
The truth is only one, scientifically studied, politically adapted, fancy
interpreted to meet the common folks needs\cite{PF}.

Objectivism failed to identify \textit{a single model} for such world: many models were designed, but none of them rules the whole be.
This is supposed to be related to the fact that even the scientists are, willing
or not, influenced by their opinion, wishes, and needs to research ``only'' particular events and properties of the world, doing that in a different way.

Basing on these facts the process that models the world and the truth has to be considered as
a combination of the two proposed models: the combination of the objective
information interpreted by the single being, influenced by the personal
experience and will, averaged in a common sense (acceptance) from the society,
and by the power that the process has on itself.
This combination varies at each time of the process progress leading to
a model that should account for an infinite factors and possibilities,
and infinite complexity. Furthermore  the formalization process itself would influence
the modelling leading to an infinite loop, thus, making the task impossible.
This process is said to be \textit{autopoiesis}.

The unmanageable complexity derived from this leads to a lack of understanding, namely
\textit{ambiguity}, that can be exploited.


\section{Aesthetics}
When dealing with \textit{ambiguity} there are also other, non mathematically demonstrated, factors that can contribute to the interpretation, and thus deciding or modelling, action.
According to McAllister the ``beauty is a sign of truth''\cite{McAllister96,McAllister98}.
Starting from this J. Morgan argues that ``beauty can play a significant role in the
``logic of pursuit''\cite{JMorgan} an \textit{a priori} style justification in contrast
with the most common \textit{a posteriori} one, the ``logic of justification''.

During the process of modelling  the unknown, ruled by the ambiguity, doing something and then
try to justify the choices made, it is not always the best option.
Exploiting the \textit{a priori} approach is thought to bring to better solutions:
the unknown needed variables could be substituted with a consideration of the
underlying (beauty) model given the choices made are
the right ones.
This approach can be found also in the decision theories that analyze
how a scientist decides what to research next, and in the considerations made
when founding a research project\cite{JMorgan}.


\section{Retrospective random reality}
Given the complexity of the model of the reality the \textit{rationality}
is said to be ``retrospective since knowledge of external constraints and forces
is insufficient to predict action and is complicated by internal differing goals
and objectives. Under these conditions, motivation is a collective process of
group decision making and ownership, restricted by any constraint from the
external environment\dots{}Under a random perspective, rationality is an emerging process wherein prediction is not foreseeable to group creates its own context''\cite{Personell}.

The modelling is so complex, mutable, and easily influenced, that the motivation can only come after the analysis, the creation of the \textit{common view}, as a justification.
Such situation is called \textit{random reality}. A reality where none has a
global understanding of it, nor leads its modification; which evolution is
subject only to the random\footnote{To be interpreted as: ``so complex that
the structure can not be even noticed''. Since nothing is truly random.}
combination of its agents.

