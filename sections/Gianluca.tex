%!TEX root = ../Chapter3.tex
\section{Conclusion}
Throughout this chapter the \textit{weak} type of ambiguity has been described like a situation in which people decide what to believe in. Everyone’s behaviour is driven by probabilities that reside inside us. In this sense it is possible to say that the whole environment is all a matter of perspective: every intention to pursue an idea is based on some beliefs, on what it is perceive as fair during our lives.

There are two main actors playing an essential role within the environment: the innovation itself and the entrepreneur.

Introducing and spreading innovation is the main objective of the entrepreneur. Doing innovation nothing is simple and it is also very different from the learning phase everyone experiences on a day-to-day basis. Innovation involves generating knowledge rather than \textit{simply} understanding it. It requires a deep exploration of the problem domain one is trying to solve and it is also a matter of dealing with other people. Some of them can be positive figures helping us achieving our goal, but some others can turn out to be discouraging and interfering with our work.
Here is where the role of the entrepreneur becomes even more critical: this kind of person is keen on exploring new horizons but, at the same time, needs to exploit what the world already offers to him.

The exploration phase is the trickiest and hardest one, since it is a common thought that news are doubtful or wrong, even though they are well supported by argumentations and facts.

Exploitation is the other key part in the innovation procedure, during which the idea is developed in a final product. It is necessary to avoid getting stuck with unobtainable dreams. The main risk is to get stuck in a circular process unless someone else invents something new. It is probably the most difficult problem the entrepreneur has to face.

Therefore, the entrepreneur’s goal is to find an equilibrium state between developing, discovering new things and consolidating what already exists. Thus, as the scholastic philosophers affirmed a long time ago, ``\textit{in medio stat virtus}'': this is the motto of the entrepreneur.

Nonetheless, the entrepreneur’s work must not be confused with reinventing something just for the sake of doing it, without giving birth to something useful, something which is really better than everything else that already exists. Nowadays, with the spread of so many startups, nearly everyone can experience the thrill of being an entrepreneur. Many people affirm to be entrepreneur of themselves, to be the head of a little company, but only few of them can say  they are developing something really innovative, something that somehow and someday can change people's habits and way of living.

The best way to deal with this continuously transforming environment is to accept and understand that only few phenomena can be fully controlled by a human being. In this sense everyone (also the innovator and the entrepreneur) has to try to find a way to accept and to deal with the randomness characterizing our inner and outer world.

A possible way out is exploiting as much as possible rationality, chasing the chance to tame the randomness characterizing the environment.

The type of ambiguity described in this chapter, that can be defined as \textit{ambiguity one}, has a key difference from the other type of ambiguity that can be found in the environment, which is presented in Chapter 4 referred to as \textit{ambiguity two}.

In the first one the entrepreneur plays the role of a \textit{speaker} supporting his ideas, presenting them to other people. He leaves the audience the freedom to choose without any pressure. On the contrary, in the latter case the entrepreneur tries to convince people of the soundness of its choices and words. Doing this way, the audience is under pressure and the freedom to believe in what they think is better tends to disappear. Manipulation takes the place of perspectives: this is where the two types of ambiguities are deeply different and how the entrepreneur’s role changes the rules of the game.
