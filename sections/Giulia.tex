%!TEX root = ../Chapter3.tex
\section{How decisions happen in an ambiguous environment: the ``garbage can model}

In an ambiguous environment, where there are no defined problems or answers, decisions cannot be constructed in a rational, hierarchical and consequential way. In this case opportunities ``happen and, sometimes, are taken as decisions according to a process identified as ``garbage can model\cite{1}\cite{2}. This model and its applications on organizations were observed and analyzed in depth by Michael Cohen, James March and Johan Olsen in 1972 \cite{1}. In that occasion, the garbage can process was defined as a mix of four streams: problems, solutions, choice opportunities and participants. Specifically, problems and solutions are associated together and related to choices, when they are in temporal proximity \cite{2}. Participants move from one choice to another, taken from the garbage can, depending on random entry time.

Problems are the result of prediction inability. For each problem, there are an entry time t (the moment in which the problem becomes visible), a quantity that defines the energy required to solve that problem, and a list of choice opportunities related to that problem. Solutions search for problems to resolve, but are distinct from them. For each solution there are an entry time t and the energy required, which represents the quantity of energy needed for that solution at moment in which the solution exits from the can and becomes part of the decision system. Choice opportunities include all the occasions in which an organization is expected to produce a decision; they  can be activated at a given time t and are eligible to a list of participants (those that are part of the organization at time t). Participants are not fixed: they can vary both in the available time they can dedicate to the decisional process at time t, both in their relative preferences, that put them in contact with a particular problem or solution instead of another one \cite{1}.

Inside the garbage can, all those elements are mixed together and then extracted randomly, in order to obtain a decision that is untied from a pre-defined preference. From this viewpoint, ``decisions do not follow an orderly process from problem to solution, but are outcomes of independent stream of events within the organization(Daft, 1982)\cite{3}. This confused state and the lack in rationality of this model can be seen as a disadvantage from a hierarchical and organized point of view. However, for an entrepreneur who is searching for innovation, it is essential not to lock ideas and, instead, to mix together multiple sources and people (each with a different garbage can of new possibilities) in order to extract something original. A real innovator never underestimates the potential of random diversity and creativity, even if it can appear as chaotic. As a quoted sentence says: ``It’s called garbage can, not garbage cannot. Even if this process can appear inefficient, it enables some type of choices that otherwise would not have been possible, because of ambiguity, lack of knowledge and order, and dynamic conflicts. In situations like those, common management strategies cannot be applied, because of the collapse of all the assumptions they base on.

The decision process that emerges from the garbage can is very interactive, with problems, solutions and participants moving from one choice to another at any time. Exploiting this feature, important choices in ambiguous organizations are often made by flight or oversight, without following a schema and purely under the control of improvisation and cooperation. Only unimportant choices are made by resolution (after some period of work) \cite{1}. This happens because important choices are those related with innovation and innovation is an art that feeds on fluidity, original ideas and ex temporary scenarios.

By describing the garbage can model from a structural point of view it is possible to identify four organizational structures, often represented through matrices or tables: the net energy load, the energy distribution, the decision structure, and the problem access structure. The net energy load is the difference between the energy required to solve all problems and the effective energy available \cite{1}, and it reflects how much the organization is ``stressed: when the value is zero it means that the energy required is equal to the energy available, yielding a case of heavy load. The energy distribution model shows the differences among all decision makers with respect to energy. In this case, it is possible to observe how the distribution of energy regarding important people in the hierarchical decision structure of the organization can change the situation of all the workers: if important people spend few energy on a problem this will have implication on the motivation of all the others. The decision structure and the access structure reflect, respectively, the possibility of connection between decision makers and choices, and the accessibility of problems with respect to choices. All those structures are strongly related with time, whose rhythm allows or impedes one situation or another.

The garbage can model has been adopted by organizations that deal with ambiguity, referring to features such as: the decision process, the energy available, and the technology exploited. In those organizations the decision process is guided more by a collection of different ideas (problems, solutions and choices mixed in the garbage can) than by a coherent structure. The production process is unclear and learns from past experience via trial and error procedures. Participants come from different domains and are characterized by differences in availability and personal preferences.

In the next section specific examples of the garbage can model applied on such organizations are presented.
