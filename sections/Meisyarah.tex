%!TEX root = ../Chapter3.tex
\section{Decision-making in an ambiguous environment according to March}

John G. March conveyed three stories of how decisions happen in organizations, which are story of decisions as intendedly rational choices, story of decisions as rule-based action, and story of decisions as artifacts\cite{2}.

\subsection{Decisions as intendedly rational choices}

Decision making is resulting from consequential and preference based action. People believe that human action is the product of human mind and choice\cite{6}. In this case, decision makers will base their choices on four things: alternatives, consequences, preferences and decision rules. They must know the alternatives for action and the consequences of each of them. Then, they must be able to put their preference order or subjective value for the alternative consequences. Independently, they have rules by which they compare and evaluate the consequences based on their preferences, in order to select suitable alternative.

There are two guesses from the theory about uncertainty in the future according to March\cite{2}. The first regards the uncertain future consequences of current action due to informational and computational limits on human choice, for example, limits on number of alternatives considered and limits on the amount and accuracy of information that is available. Information has to be discovered through search. By evaluating the performance to achieve goals, the search is continued until it discloses an alternative that seems good enough to satisfy objectives. In this way, the new alternative is added to the existing options.

The second guess is the uncertain future preferences for consequences of current action. According to theories of choice\cite{2}, preferences have the following properties: subjectively absolute, stable, consistent, precise, and exogenous. In fact, the practice and the theory are not consistent. Decision makers often make decisions based on rules and tradition, instead of fully conscious preferences. Preferences are inconsistent and change over time, so that predicting future preferences is difficult since the uncertainty appears. Furthermore, it seems like actions and their consequences affect preferences, although preferences, in theory, are used to select an action. However, such differences described in theories and in actual appearance can be interpreted as reflecting some ordinary behavioral wisdom of human beings, which does not always parallel the theory. Even though human beings want consistency, inconsistency shows them necessary aspects of the development and clarification of preferences. When inconsistency appears, the preferences to make decisions become ambiguous\cite{6}.

Another issue in rational choice model that has to be noticed is risk taking. Decision makers often face a situation where they can consider it as opportunity or danger. Risk always sticks on any decision or alternative or step that human beings take. It is not easy to estimate the probability of the risk, to decide whether to take the risk or to avoid it. Although decision makers can make sure the probability that something happens is between 0 and 1, they cannot be sure whether the probability of them taking the risk is high or low, a good or bad option. They face the ambiguity, in which the probability is unknown and it is hard to say their decision is right or not. Even when they decide to avoid risk, inadvertently, they take the risk.

Decision makers must be able to distinguish between taking risk and gambling. When they find variability, they try to control it rather than just treat it as chain to expected value in making a choice\cite{7}. And in general, experiences become the biased factor for decision makers in risk taking. They ‘believe’ that they will do rather well in estimating future probabilities in situations in which they have experience. Past successes raise self-confidence in handling future probabilities and leading them to believe their own ability and insight\cite{2}. In contrary, the successes make them hardly believe in the existence of luck. Consequently, the successful decision makers tend to underestimate the risk they have through experience and the risk they currently face\cite{7}. When facing uncertainty and ambiguity, decision makers assume that uncertainty can be removed and they tend to try improving the quality of information. Also, they tend to trust few signs and exclude the rest from consideration when inconsistent information appears\cite{8}.

\subsection{Decisions as rule-based action}

This vision exposes how action is the product of tradition, rule, routine, or revelation. Rather than by evaluating alternatives based on the values of consequences, decision making often rises by finding appropriate rules to follow\cite{2}. This model involves three factors: situation, identity, and matching. The rules of appropriateness match the situation and identities. Since it does not calculate the future consequences of current action, the changing or development of the rules is needed to match the changing situations. And these ‘changings’ are often ambiguous. When decision makers are uncertain with their own utility function, preference ambiguity is present\cite{9}.

Rules can be seen as an implicit agreement among rational parties to act appropriately in return for being treated appropriately, and current rules contain information produced by experience\cite{2}. Rules can be modified as the organization learn from experience, for example, on the basis of feedback from the environment. The rule development can also be done by selecting rules that produce optimal decisions out of collections of invariant rules. Moreover, it is possible that decision makers copy each other and this is a common feature of ordinary organizational adaptation. The action of rules modification, selection, or imitation, can be a possible change that decision makers might do as innovators. However, each of them does not always work in any case. There are some factors like rates of change, consistency, and foolishness that decision makers should consider.

\subsection{Decisions as artifacts}

The two models described above seem to treat the outcomes as the main product of the decision process. In fact, the environment can be more complex and confusing. Many things are happening at once: technologies are changing; preferences and perceptions are changing; problems, solutions, ideas, people and even outcomes are mixed together making their interpretation uncertain and their connection unclear\cite{2}. Thus, there is a change in the idea of how the connection between decisions and decision making is tought. March conveyed the new concept of decisions and decision making as artifacts emphasizes network of linkages within and among organizations rather than hierarchies, temporal orders rather than causal orders, loose coupling between decisions and decision making rather than tight coupling, and the role of decisions and decision making in the development of meaning and interpretation.

By emphasizing temporal orders in organizational decision making, decisions can be seen as the consequence of combining different events involving different individuals. The idea is that it is not possible that individuals or groups simultaneously involve in different event. If individuals attend to some things, they will not attend the others. Thus, the attention is engaged by certain participants to certain decision. And it depends on alternative claims on attention\cite{2}. Such idea is a generalization that leads to a decision making model called ‘garbage can model’ which will be deeply explained in the next part. Due to their temporal proximity, problems and solutions are attached to each other, and to the choices as well that can lead to the decision. The meaning constructed by this alternative idea (not theory of choice) is that life is not mainly a choice, it is an interpretation, and process gives more substantial meaning to life than outcome.

