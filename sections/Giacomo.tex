%!TEX root = ../Chapter3.tex
\section{spaghetti organizations}
In the previous section the garbage can model has been described: it depicts an organizational structure in which decisions are made under an uncertain environment. As stated in\cite{4}, decision-making is something that happens in presence of limited rationality, where organizations, and people that participate to the decisional process, learn from their experience. This puts the basis for future actions, in a continuous and cyclic process called ``adaptively rational''. In that sense the garbage can model highlights how individuals and organizations try to make sense of their experience, and consequently modify their behaviour in light of their interpretation of events. Organizations make experience, but without a specific structure or method that drives it.

This section gives two examples of organizations classes in which decision-making faces situation that fully represent the garbage can model presented before: unclear goals, technologies and fluid participants. Due to this, it is possible to refer to them as \textit{spaghetti organization}, to underline the aspects of randomness and intricacies of their structure.

The first class presented are the \textit{universities}, as discussed in\cite{1}. Choices are typically made without attention to existing problems and with minimum effort in terms of time and energy (oversight). Moreover, problems pass repeatedly from a choice to another, but decisions resolve no problems (flight), such that, ``it’s likely that important choices don’t solve problems''. This ideal \textit{garbage can}, made of problems, choices, and decision makers, rearranging themselves, makes the understanding of both solutions and choices increasingly difficult, and this is the reason why the decisional process is susceptible to an increase in the work-load.

Even when problems are solved, by oversight or flight, choices are typically not the first problems they were attached to. In such scenario, a right matching between the \textit{garbage can} actors leading to a solution is also dependant from timing and the current load of the system.
An analysis of the response of the model when an advert event happens follows.

A particular event is called \textit{slack}, and is defined as the excess of capacity maintained by the organization. It emerges when resources are not fully allocated and organizations do not completely utilize them. This situation can determine episodes in which, for example, ``employees are paid more than necessary to retain them within the organization and the firm is charging its products less than it would be possible''\cite{13}. The example before suggests that \textit{slack} is sensible to both the external resources provided to the organization and the consistency of internal demand due to its participants.

The \textit{reduction} of \textit{slack} represents an advert event inside the organization, and implies some important consequences in the garbage can decisional process. This scenario has been analysed by Cohen, March and Olsen in 1972\cite{1} taking into account four parameters that allow to infer the response of the structure previously described in universities and schools: \textit{net energy load}; \textit{access structure}; \textit{decision structure} and \textit{energy distribution}. A part of their experiment is now reported, in order to clarify the university organization. Four types of universities have been considered: (a) large, rich universities, (b) large, poor universities, (c) small, rich universities and (d) small, poor universities. Given such pattern, relatively rich universities are expected to have a light energy load and an unsegmented decision structure, representing a particular rigid and hierarchical scenario, whereas poor ones show a different energy distribution depending on the structure dimension. Moreover, the amount of unresolved problems (under the hypothesis of \textit{slack}) will be higher in poor institution than rich ones. Table~\ref{tab:a} summarises this initial scenario.

\begin{table}
% h!
\centering
\caption{Results in pattern variation under slack reduction.}
\label{tab:a}
    \begin{tabular}{lcccc}
        \toprule
            & LOAD & ACCESS STRUCTURE & DECISION STRUCTURE & ENERGY DISTRIBUTION \\
        \midrule
            LARGE, RICH & Light & Specialized & Unsegmented & Less \\
            LARGE, POOR & Moderate & Hierarchical & Hierarchical & More \\
            SMALL, RICH & Light & Unsegmented & Unsegmented & More \\
            SMALL, POOR & Moderate & Specialized & Specialized & Equal \\
        \bottomrule
\end{tabular}
\end{table}

The garbage can model can be used to predict the variation of these parameters under \textit{slack reduction}. Availability of resources will be shorter, thus the increase of net energy load, because problems need more effort to be solved. Adversity also affects both access structure and decision structure and shifted toward, respectively, a more hierarchical access and decision structure, because administrators are less able to move from one decision to another. However, poor universities are in the worst position under adversity, due to a much higher increase than rich institutes in problem activity and decision time. The model also predicts that any institution will experience improvements in their position over longer times.

Another example of \textit{spaghetti organization}, analysed in\cite{5}, is given by a danish firm, Oticon, producer and market-leader of ear aids, that in early 1990’s decided to operate a radical organization transformation, representing a dramatic change from the previous ``hierarchical, functional-based organization''.

The adoption of this model was due to the loss of competitive advantages of its products that were ``largely dependent from a mature and declining technology'' and aimed to create a more creative and entrepreneurial organization structure. When the \textit{spaghetti organization} was introduced it immediately produced substantial technological innovation during the 1990’s.
However it was gradually abandoned due to lack of ``coordination and problem sharing within projects''. In the end, the management attempted to selectively interfere in projects (\textit{selective-invention}), causing an increasing frustration among the employees.
